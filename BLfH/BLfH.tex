\documentclass[a4paper]{article}

\title{Bastard Language from Hell: What is wrong with SINGULAR}
\begin{document}
\maketitle

In this document we collect questions about the SINGULAR programming
language. 

\section{Ring Maps}
\label{sec:ring-maps}

\begin{description}
\item[map] Defines a ring map by giving the images of
  variables. Is standard
\item[imap] Defines a ring map which preserves the names of variables.
\item[fetch] Defines a ring map which preserves order of variables.
\end{description}

\section{Some useful commands}
\label{sec:some-useful-commands}

\begin{description}
\item[kill] unassigns a variable
\end{description}

\section{Ideal vs. Standard Bases}
\label{sec:ideal-vs.-standard}

One major mistake when designing Singular was not to implement
seperate types for ideal and standard bases. Many algorithms require
standard bases as input but will still run (and produce wrong results)
if called with an ideal as argument. 

\section{Random usage of suprising names: Case Study: minpoly}
\label{sec:minpoly}
In Singular the word ``minpoly'' is a keyword which is used in the
definition of rings. The info page says :

`*Purpose:*'
     describes the coefficient field of the current basering as an
     algebraic extension with the minimal polynomial equal to `minpoly'.
     Setting the `minpoly' should be the first command after defining
     the ring.

I think, this is just a crazy design mistake.


\end{document}

%%% Local Variables: 
%%% mode: latex
%%% TeX-master: t
%%% End: 
